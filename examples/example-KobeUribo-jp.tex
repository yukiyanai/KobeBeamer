\documentclass[dvipdfmx]{beamer}
\usepackage{amssymb,amsmath}
\usepackage{graphicx}
\usepackage{booktabs}
\usepackage{dcolumn}

\usepackage{mathptmx}
\usepackage{helvet}
\usepackage{courier}

%% 日本語設定
\usepackage[deluxe,expert]{otf}
\renewcommand{\kanjifamilydefault}{mg}
\renewcommand{\figurename}{図}
\renewcommand{\tablename}{表}


%% しおりの文字化けを回避する
\usepackage{atbegshi}
%\ifnum 42146=\euc"A4A2 \AtBeginShipoutFirst{\special{pdf:tounicode EUC-UCS2}}
%  \else \AtBeginShipoutFirst{\special{pdf:tounicode 90ms-RKSJ-UCS2}}
%\fi

\usetheme[jp,simplefoot]{Kobe}

\title[KobeBeamer]{神戸大学非公式 Beamer テーマ}
\subtitle{神戸スタイルの {\LaTeX} プレゼンテーション}
\author{矢内 勇生}
\institute{法学研究科}
\date{2015年4月20日}

\begin{document}

\begin{frame}
 \maketitle
 %% Uncomment the following to remove header and footer from title page
 %\thispagestyle{empty}
\end{frame}

\begin{frame}{内容} 
  \tableofcontents
\end{frame}

\section{はじめに}
\subsection{神戸大学用 Beamer テーマ}

\begin{frame}{KobeBeamerでスライドを作ろう!}
 神戸大学の公式ロゴで使用されている色:
  \begin{itemize}
    \item \textcolor{kobebrick}{ブリック}:神戸大学のシンボル
    \item \textcolor{kobegreen}カラー{グリーン}:山のイメージ
    \item \textcolor{kobeblue}{ブルー} :海のイメージ
    \item \textcolor{kobegray}{グレー} :文字に使用
  \end{itemize}
\end{frame}



\section{基本}


\subsection{ブロック}

\begin{frame}{ブロックを使おう}
  \begin{block}{Block}
   これは block 環境です。
  \end{block}
  \pause
  \begin{exampleblock}{Example}
   これは example block 環境です。 
  \end{exampleblock}
 \pause
  \begin{alertblock}{Alert}
   これは alert block 環境です。
  \end{alertblock}
\end{frame}


\subsection{数式}

\begin{frame}{数式を使おう}
 正規分布 $\mathrm{N}(\mu, \sigma^2)$ の確率密度関数: 
 \begin{equation}
  f(x) = \frac{1}{\sqrt{2 \pi \sigma^2}}\exp\left[-\frac{(x - \mu)^2}{2 \sigma^2} \right]
 \end{equation}  

\begin{block}{標準正規分布の確率密度関数}
  \begin{equation}
   f(x) = \frac{1}{\sqrt{2 \pi}}\exp\left(-\frac{x^2}{2}\right)
  \end{equation}
\end{block}
 \end{frame}


\section{図表}

\subsection{表} 
 
\begin{frame}{表で説明する}
  \begin{table}
   \vspace{-18pt}
   \caption{OLSによる推定結果:応答変数は得票率 (\%)}
    \begin{tabular}{lD{.}{.}{-2}D{.}{.}{-2}}
     \hline
                           & \multicolumn{2}{c}{推定値}\\
     説明変数 & \multicolumn{1}{c}{モデル1} &
     \multicolumn{1}{c}{モデル2} \\
     \hline
     定数   & 7.91 & -2.07\\
                & (0.69) & (0.72)\\
     議員経験 & 18.10  & 45.91\\
                & (1.23) & (1.58)\\
     選挙費用    & 1.85   & 4.87\\
                & (0.12) & (0.16)\\
     議員経験 $\times$ 選挙費用 &  & -4.76\\
                                 &  & (0.21)\\ 
     \hline
     サンプルサイズ($n$) & \multicolumn{1}{c}{1124} & \multicolumn{1}{c}{1124}\\
     自由度調整済み $R^2$ & 0.56 & 0.70\\
     \hline 
     \multicolumn{3}{l}{\footnotesize 注:括弧内は標準誤差}\\
    \end{tabular}
  \end{table}

\end{frame}

\subsection{図}

{
\setbeamertemplate{footline}{}
\frame{{図で説明する}
 \begin{figure}[h]
   \centering
   \includegraphics[scale=.9]{normal-density.pdf}
   \vspace{-12pt}
   \caption{正規分布のPDF}
 \end{figure}
}}


\begin{frame}{写真}
  \begin{columns}[t]
    \begin{column}{0.5\textwidth}
     \centering
     \includegraphics[scale=.4]{figs/bayes.jpg}\\
     Thomas Bayes
   \end{column}
    \begin{column}{0.5\textwidth}
     \centering
     \includegraphics[scale=.7]{figs/laplace.jpg}\\
     Pierre-Simon Laplace
   \end{column}
 \end{columns}
 \vspace{12pt}
 \[
  p(\theta | y) = \frac{p(y | \theta)p(\theta)}{p(y)}
 \]
\end{frame}


\section{まとめ}
\subsection{Let's use KobeBeamer!}

\begin{frame}{結論}
 \LaTeX と KobeBeamer を使えば
 \begin{itemize}
  \item 美しいスライドが作れる!
  \item \alert{神戸大学} のイメージを世界にアピールできる!
 \end{itemize}
 \pause
 \vspace{36pt}
 
 \hspace{2em} Email: 
 \href{mailto:yanai@lion.kobe-u.ac.jp}{\texttt{yanai@lion.kobe-u.ac.jp}}
 \end{frame}

\end{document}